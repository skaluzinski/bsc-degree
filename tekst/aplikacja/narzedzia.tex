\section{Lista użytych technologii}
\subsubsection{Kotlin}\label{app:used_technologies:kotlin}
Aplikacja mobilna została w całości zaimplementowana z pomocą języka programowania Kotlin.
Kotlin został oficjalnie wybrany przez Google jako preferowany język programowania dla aplikacji mobilnych na platformę Android 7 maja 2019 roku.
\cite{kotlinWikipedia}
W przypadku aplikacji mobilnej wybranie Javy zamiast Kotlina skutkowałoby w byciu zmuszonym do imperatywnego frameworka do tworzenia widoków.
Java nie pozwala na użycie deklaratywnego frameworka oraz ten język nie jest już rozwijany przez Google od wielu lat.
Z tego powodu wybranie Javy skutkowałoby w przyszłym "tech debt".
W momencie pisania aplikacji tworzylibyśmy kod, który staje się obciążeniem i problemem w przyszłości od momentu wytworzenia go.
Użycie technologii, która nie jest rozwijana i będzie wymagała przepisania możemy uznać za źródło długu wdrożeniowego.
\cite{legacyDebt}
Rozwijanie nowego projektu z już obecnym długiem technologicznym jest równoznaczne z zaciągnięciem długu technologicznego, który będzie musiał zostać spłacony w przyszłości.
Dług technologiczny (ang. technical debt) jest terminem używanym do opisania różnicy między aktualnym stanem naszego rozwiązania a idealnym stanem, do którego powinno się dążyć w celu uzyskania skalowanego produktu, który jest łatwy w utrzymaniu.
\cite{techDebtAccerelate}

\subsection{Compose}\label{app:used_technologies:compose}
Compose to deklaratywny framework do tworzenia interfejsów użytkownika w języku Kotlin.
Deklaratywność pozwala skupić się programiście na opisaniu zachowania interfejsu zamiast skupiania się na manipulacji samym wyglądem interfejsu.
W przeciwieństwie do imperatywnego podejścia, które było preferowanym sposobu tworzenia interfejsu przez ponad 13 lat, compose wymaga odpowiedniego oddzielenia stanu od widoku.
Framework ten działa na podstawie tworzenia drzewa widoków, które jest renderowane na ekranie.
\cite{composeTree}
W momencie kiedy zmienia się wartość przekazywana jako parametr do widoku, framework automatycznie przerysowuje widok.
Sytuację tą nazywamy rekompozycją.
\cite{composeRecomposition}
Framework decyduje czy należy wykonać rekompozycję na podstawie zmiany wartości przekazywanej do widoku, która jest sprawdzana za pomocą metody \texttt{equals}.
Dlatego ważne jest aby przekazywać do widoków jak najprostsze obiekty.
Każde porównanie wartości przekazywanych ma swój koszt obliczeniowy.
Im bardziej złożony obiekt przekazujemy do widoku, tym więcej czasu zajmie porównanie go z poprzednią wartością.
Pojedyńcze obiekty stanowiłyby problemu, jednak w sytuacji gdy nasza aplikacja miałaby dziesiątki widoków, które musiałyby sprawdzać obiekty będące boskimi obiektami (ang. god objects), mogłoby to spowodować spadek wydajności aplikacji.
\cite{godObjectWikipedia}
Wymusza to na programiście odpowiednie rozdzielenie stanu danej części aplikacji od innych części, przekazywanie wcześniej wymienionych boskich obiektów do rożnych części aplikacji powodoałoby konieczność rekompozycji całej aplikacji.
Można to zoobrazować na przykładzie sytuacji kiedy zmiana wpisanego tekstu w formularzu powoduje konieczność ponownego obliczenia całego drzewa widoków.
Dodatkowo framework ten opiera się na paradygmacie kompozycji zamiast dziedziczenia.
W przypadku androida, dziedziczenie widoków było jednym z największych problemów, które wraz z czasem urosły do nieprzewidywalnych rozmiarów.
Każdy widok dziedziczył po klasie \texttt{View}, co powodowało, że każdy widok również dziedziczył problemy występujące pośród 40 tysięcy linii kodu.
Wraz z czasem pojawiały się problemy, które zostały zignorowane przez firmy tworzące swoje nakładki na androida.
Rozwiązanie problemów występujące na oryginalnej platformie w połączeniu z problemami występującymi na nakładkach powodowały, że systemy stawały się coraz bardziej niestabilne i trudniejsze w utrzymaniu.
Aby zaimplementować prostą listę elementów w XMLu, programista musiał zaimplementować dziedziczenie po klasie \texttt{ListView}, która dziedziczyła po klasie \texttt{ViewGroup}, która dziedziczyła po klasie \texttt{View}.
Dodatkowo trzeba stworzyć odpowiedni adapter, który będzie odpowiedzialny za przekazywanie danych do widoku.
Finalnie trzeba przypisać adapter do widoku, zaaktualizować widok i przekazać dane do adaptera.
Przykład widoku:
\begin{lstlisting}[language=xml]
    <FrameLayout
        android:layout_width="match_parent"
        android:layout_height="match_parent">
        <androidx.recyclerview.widget.RecyclerView
    android:id="@+id/recyclerView"
    android:layout_width="match_parent"
    android:layout_height="match_parent"
    tools:listitem="@layout/item_layout" />
    </FrameLayout>
\end{lstlisting}
Przykład implementacji pojedyńczego widoku w liście:
\begin{lstlisting}[language=xml]
<TextView
    android:id="@+id/itemText"
    android:layout_width="match_parent"
    android:layout_height="wrap_content"
    android:padding="16dp"
    android:textSize="16sp" />
\end{lstlisting}
Przykład implementacji adaptera:
\begin{lstlisting}[language=Kotlin]
class ItemAdapter(private val items: List<Item>) : RecyclerView.Adapter<ItemAdapter.ItemViewHolder>() {
    override fun onCreateViewHolder(parent: ViewGroup, viewType: Int): ItemViewHolder {
        val view = LayoutInflater.from(parent.context).inflate(R.layout.item_layout, parent, false)
        return ItemViewHolder(view)
    }

    override fun onBindViewHolder(holder: ItemViewHolder, position: Int) {
        holder.bind(items[position])
    }

    override fun getItemCount(): Int = items.size

    class ItemViewHolder(itemView: View) : RecyclerView.ViewHolder(itemView) {
        private val itemText: TextView = itemView.findViewById(R.id.itemText)

        fun bind(item: Item) {
            itemText.text = item.text
        }
    }
}
\end{lstlisting}
Finalnie należy przypisać adapter do widoku:
\begin{lstlisting}[language=Kotlin]
class MainActivity : AppCompatActivity() {
    override fun onCreate(savedInstanceState: Bundle?) {
        super.onCreate(savedInstanceState)
        setContentView(R.layout.activity_main)

        val recyclerView: RecyclerView = findViewById(R.id.recyclerView)
        recyclerView.layoutManager = LinearLayoutManager(this)

        val items = listOf(Item("Item 1"), Item("Item 2"), Item("Item 3"))
        recyclerView.adapter = ItemAdapter(items)
    }
}
\end{lstlisting}
Równoznaczny kod w compose do utworzenia widoku
\begin{lstlisting}[language=Kotlin]
@Composable
fun ItemList(
    myItems: List<Item>
) {
    Column {
        myItems.forEach { item ->
            Text(text = item.text)
        }
    }
}
\end{lstlisting}
Przypisanie widoku do aktywności aplikacji:
\begin{lstlisting}[language=Kotlin]
class MainActivity : AppCompatActivity() {
    override fun onCreate(savedInstanceState: Bundle?) {
        super.onCreate(savedInstanceState)
        setContent {
            val items = listOf(Item("Item 1"), Item("Item 2"), Item("Item 3"))
            ItemList(items)
        }
    }
}
\end{lstlisting}
Dodatkowym plusem jest to, że całośc rozwoju aplikacji odbywa się w jednym języku programowania.
Sztuczna inteligencja w IDE pozwala na szybkie i sprawne podpowiedzi, które znacznie przyspieszają proces tworzenia aplikacji.
\subsection{Koin}\label{app:used_technologies:koin}
Do wstrzykiwania zależności w aplikacji został użyty framework Koin.
Jest on biblioteką, która pozwala na wstrzykiwanie zależności w sposób deklaratywny, dodatkowo jest możliwa do wykorzystania w przypadku aplikacji androidowych, backendowych, desktopowych oraz wieloplatformowych.
Pozwala to na przygotowanie aplikacji, która może być łatwo przenoszona między różnymi platformami.
Dzięki temu przygotowaniu wystarczyłoby zmienić implementację danego interfejsu odpowiednio dla platformy, a cała reszta aplikacji pozostałaby bez zmian.
\cite{koinMultiplatform}

\subsubsection{Ktor}\label{app:used_technologies:ktor}
Ktor jest biblioteką do tworzenia serwerów HTTP w języku Kotlin oraz klientów HTTP.
Użycie tej biblioteki pozwala uniknąć potrzeby przepisywania kodu w sytuacji chęci uczynienia z aplikacji natywnej aplikację wieloplatformową.
Pozwala to na wykorzystanie napisanego kodu na aplikację androidową jako wspólny kod dla aplikacji wieloplatformowej.
\cite{ktorMultiplatform}
