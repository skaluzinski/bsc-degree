\section{Problemy}
W czasie implementacji serwera napotkałem wiele problemów wynikających z obranych wyborów pod względem zarówno architektury jak i technologii.
Z powodu wybrania wielomodularnej architektury problematyczne okazało się wykorzystanie wzajemnie modułów przez siebie.
Był to częściowo zamierzany efekt, oczekiwałem, że wydzielenie odpowiednich domen utrudni zignorowanie ich granic.
Problemem okazała się autoryzacja użytkownikam, a dokładnie inaczej rozwijające się potrzeby modułów.
Wraz z czasem implementacji dochodziłem do wniosku, że nie chciałbym aby zwyczajny użytkownik mógł zostać "nauczycielem" bez odpowiedniej weryfikacji.
Wymóg ten pojawił się z powodu problemów, które mogą wystąpić gdy zarejestrowane osoby mogłyby udostępniać nieodpowiednie treści.
Jako twórca systemu ponoszę odpowiedzialność za umieszczane w niej treści oraz jej sposób wykorzystywania.
Przemyślenia te spowodowały kompletnego logicznego oddzielenia rejestracji ucznia od nauczyciela.
Dodatkowo uznałem, że jedna osoba mogłaby chcieć używać aplikacji zarówno jako uczeń oraz nauczyciel.
Z tego powodu rozdzieliłem całkowicie te funkcjonalności, użytkownik może stworzyć zarówno konto nauczycielskie jak i uczniowskie używając tych samych danych.

Wstrzykiwanie zależności w spring bootcie okazało się nie być przystosowane do mojego sposobu implementacji.
Definicja interfejsów w pod modułach oraz ich przypisanie w głównym, współdzielonym module powoduje, że graf zależności spring boota nie potrafi znaleźć odpowiednich beanów.
Są one odpowiednie znajdowane w runtime, niestety środowisko programistyczne nie jest w stanie ich interpretować.

Dodatkowo pojawiły się problemy powiązane z spring security, niestety część tych funkcjonalności wymaga konkretnego użycia.
Sama dokumentacja jest myląca z tego powodu jak wiele zmieniło się z biegiem czasu, w nowszych wersjach tej biblioteki większośc sposobów użycia jest już przestarzałe.
Niestety większość poradników jak i artykułów pokazuje przestarzałe sposoby implementacji, które nie są już wspierane.
Również samo użycie spring-security do pomocy w zakresie weryfikacji sesji powodowałoby wymóg wprowadzenia ogromnych zmian w strukturze aplikacji.
Z tego powodu mechanizm sesji jak i redis został finalnie usunięty z projektu.
Tranzystywny sposób w jaki zaimplementowane przezemnie klasy zależały od spring-security jak i odwrotnie spodował, że w aplikacji pojawiły się cykliczne zależności.
Był to jeden z problemów, które chciałem ominąć poprzez zastosowane modularnej architektury.
