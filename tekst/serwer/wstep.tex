W celu zaimplementowania serwera backendowego wykorzystano technologię Spring Boot.
Spring boot umożliwia szybkie tworzenie aplikacji w języku Java (jak i Kotlin), a także zapewnia wiele gotowych rozwiązań, które znacznie ułatwiają pracę.
Sprint Boot opiera się na frameworku Spring, który jest jednym z najpopularniejszych frameworków do tworzenia aplikacji backendowych.
Sam Spring wymagałby znacznie wiecje konfiguracji, co znacznie wydłużyłoby czas implementacji.
Użycie Spring Boot pozwala na skupienie się na implementacji funkcjonalności, a nie na konfiguracji,
uzycie Spring zamiast Spring Boot wymagałoby znacznie wiekszego nakładu pracy ale byłoby równie możliwe.

