\section*{Cel pracy}
Głównym celem pracy "Implementacja aplikacji mobilnej do organizacji korepetycji" jest stworzenie aplikacji mobilnej,
która pozwoli na odpowiednia skalowanosc systemu, aby moc z latwością dodawać nowe funkcjonalności.
W przypadku serwera backendowego, celem jest stworzenie systemu, który pozwoli na zastaoswanie w przyszłości takich technologii jak mikroserwisy.
Aby zrealizować ten cel zostanie zastosowana wielo-modularna architektura, która pozwoli na latwa podmiane modułów w przyszłości na odpowiednie mikroserwisy.

W przypadku aplikacji mobilnej, celem jest stworzenie aplikacji, która pozwoli na odpowiednie odseparowanie warstw od siebie.
Będzie to osiągnięte poprzez wydzielenie odpowiednio agnostycznych interfejsów względem platformy, na której działa aplikacja.
Jest to ważne, ponieważ docelowo aplikacja ma być dostępna na systemach Android oraz iOS. Zostanie to osiągnięte poprzez zastosowanie takiej technologii jak Kotlin Multiplatform.
Dlatego niezwykle ważne jest, aby aplikacja w należyty sposób odseparowywała warstwy od siebie, aby w przyszłości móc z łatwością zastąpić warstwę widoku na odpowiednią dla systemu iOS.
