\section*{Wprowadzenie}
W wyniku dynamicznej cyfryzacji życia, przyspieszonej przez pandemię, na rynku korepetycji pojawiła się wyraźna luka.
Obecnie dostępne rozwiązania nie spełniają potrzeb użytkowników pod względem wygody zarządzania zajęciami i jakości oferowanych usług.
Brakuje narzędzia, które umożliwiłoby kompleksowe zarządzanie harmonogramem,płatnościami i komunikacją pomiędzy uczniem a korepetytorem,
co stwarza zapotrzebowanie na nowoczesną, dedykowaną aplikację.

Praca "Aplikacja mobilna do organizacji korepetycji" podejmuje implementacja podwalin do rozwijania skalowanej aplikacji mobilnej, która umożliwi uzupełnienie tej luki na powstałe na rynku.
W ramach stworzenia aplikacji zostanie zaimplementowanie zaimplementowana natywna aplikacja na system Android oraz serwer backendowy w technologii Spring Boot.