Przed jakimkolwiek rozpoczęciem prac na projektem podstanowiłem odpowiednio zdefiniować zarys systemu jaki chciałem stworzyć.
Postanowiłem odpowiednio rozrysować przypadki użycia, które chciałbym obsłużyć.
Wypisanie przypadków użycia nie ma służyć do określenia wymogów funkjconalnych tej pracy, ma jedynie służyć do zrozumienia zachowania oraz wzięcia pod uwagę potrzeb przyszłościowych pod wzlgędem rozwoju.
Wszelkie diagramy zostały wykonane w technologii plantUML.
Wybór ten pochodził z faktu, że używam tej technologii w pracy komercyjnej jako programista do tworzenia schematów architektury projektu.
Technologia ta pozwala na tworzenie diagramów w formie tekstowej.
Jest to niezwykle przydatne w dobie sztucznej inteligencji oraz automatyzacji.
Możliwe byłoby proste stworzenie skryptu, który na podstawie pliku z diagramem wygenerowałby kod w dowolnym języku programowania.
Również proste byłoby stworzenie przeciwnego skryptu, który na podstawie kodu wygenerowałby diagram.


